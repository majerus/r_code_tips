\documentclass[]{article}
\usepackage[T1]{fontenc}
\usepackage{lmodern}
\usepackage{amssymb,amsmath}
\usepackage{ifxetex,ifluatex}
\usepackage{fixltx2e} % provides \textsubscript
% use upquote if available, for straight quotes in verbatim environments
\IfFileExists{upquote.sty}{\usepackage{upquote}}{}
\ifnum 0\ifxetex 1\fi\ifluatex 1\fi=0 % if pdftex
  \usepackage[utf8]{inputenc}
\else % if luatex or xelatex
  \ifxetex
    \usepackage{mathspec}
    \usepackage{xltxtra,xunicode}
  \else
    \usepackage{fontspec}
  \fi
  \defaultfontfeatures{Mapping=tex-text,Scale=MatchLowercase}
  \newcommand{\euro}{€}
\fi
% use microtype if available
\IfFileExists{microtype.sty}{\usepackage{microtype}}{}
\usepackage[margin=1in]{geometry}
\usepackage{graphicx}
% Redefine \includegraphics so that, unless explicit options are
% given, the image width will not exceed the width of the page.
% Images get their normal width if they fit onto the page, but
% are scaled down if they would overflow the margins.
\makeatletter
\def\ScaleIfNeeded{%
  \ifdim\Gin@nat@width>\linewidth
    \linewidth
  \else
    \Gin@nat@width
  \fi
}
\makeatother
\let\Oldincludegraphics\includegraphics
{%
 \catcode`\@=11\relax%
 \gdef\includegraphics{\@ifnextchar[{\Oldincludegraphics}{\Oldincludegraphics[width=\ScaleIfNeeded]}}%
}%
\ifxetex
  \usepackage[setpagesize=false, % page size defined by xetex
              unicode=false, % unicode breaks when used with xetex
              xetex]{hyperref}
\else
  \usepackage[unicode=true]{hyperref}
\fi
\hypersetup{breaklinks=true,
            bookmarks=true,
            pdfauthor={Rich Majerus},
            pdftitle={Reproducible Research},
            colorlinks=true,
            citecolor=blue,
            urlcolor=blue,
            linkcolor=magenta,
            pdfborder={0 0 0}}
\urlstyle{same}  % don't use monospace font for urls
\setlength{\parindent}{0pt}
\setlength{\parskip}{6pt plus 2pt minus 1pt}
\setlength{\emergencystretch}{3em}  % prevent overfull lines
\setcounter{secnumdepth}{0}

\title{Reproducible Research}
\author{Rich Majerus}
\date{August 27, 2014}

\begin{document}

\begin{center}
\huge Reproducible Research \\[0.2cm]
\large \emph{Rich Majerus}\\[0.1cm]
\large \emph{August 27, 2014} \\
\normalsize
\end{center}


\begin{verbatim}
## % latex table generated in R 3.1.0 by xtable 1.7-3 package
## % Thu Aug 28 09:13:08 2014
## \begin{table}[ht]
## \centering
## \begin{tabular}{rlrrrrrrr}
##   \hline
##  & State & 2007 & 2008 & 2009 & 2010 & 2011 & 2012 & 2013 \\ 
##   \hline
## 1 & alabama &   0 &   5 &   1 &   0 &   0 &   0 &   1 \\ 
##   2 & alaska &   2 &   0 &   3 &   1 &   3 &   3 &   2 \\ 
##   3 & arizona &  12 &   7 &   3 &   3 &   5 &   8 &   5 \\ 
##   4 & arkansas &   0 &   0 &   0 &   2 &   0 &   0 &   0 \\ 
##   5 & california &  71 &  65 &  94 &  96 &  97 &  85 &  87 \\ 
##   6 & colorado &   8 &  14 &   5 &  11 &  13 &   7 &   7 \\ 
##   7 & connecticut &   5 &   7 &   5 &   9 &  13 &   7 &   3 \\ 
##   8 & delaware &   0 &   1 &   1 &   0 &   0 &   0 &   0 \\ 
##   10 & florida &   8 &   9 &   5 &  10 &   4 &   4 &   9 \\ 
##   11 & georgia &   1 &   7 &   4 &   3 &   7 &   2 &   0 \\ 
##   12 & hawaii &   5 &   1 &   3 &   1 &   2 &   2 &   3 \\ 
##   13 & idaho &   1 &   2 &   2 &   3 &   3 &   2 &   4 \\ 
##   14 & illinois &  12 &   5 &   5 &  12 &   4 &   3 &  11 \\ 
##   15 & indiana &   1 &   2 &   2 &   3 &   2 &   1 &   0 \\ 
##   16 & iowa &   1 &   1 &   2 &   2 &   1 &   2 &   2 \\ 
##   17 & kansas &   0 &   0 &   1 &   1 &   1 &   0 &   0 \\ 
##   18 & kentucky &   0 &   0 &   1 &   0 &   0 &   1 &   0 \\ 
##   19 & louisiana &   0 &   4 &   1 &   2 &   0 &   2 &   3 \\ 
##   20 & maine &   2 &   1 &   2 &   3 &   4 &   2 &   2 \\ 
##   21 & maryland &   5 &   3 &   7 &   7 &   2 &   5 &   5 \\ 
##   22 & massachusetts &  15 &  20 &  23 &  18 &  19 &  15 &  17 \\ 
##   23 & michigan &   3 &   2 &   4 &   2 &   3 &   3 &   2 \\ 
##   24 & minnesota &   9 &  11 &   9 &   6 &   6 &   2 &   7 \\ 
##   25 & mississippi &   0 &   0 &   0 &   0 &   0 &   1 &   0 \\ 
##   26 & missouri &   5 &   2 &   7 &   4 &   3 &   5 &   0 \\ 
##   27 & montana &   2 &   1 &   1 &   2 &   1 &   1 &   1 \\ 
##   28 & nebraska &   1 &   1 &   0 &   1 &   1 &   0 &   0 \\ 
##   29 & nevada &   3 &   0 &   2 &   0 &   2 &   3 &   1 \\ 
##   30 & new hampshire &   4 &   1 &   5 &   3 &   2 &   6 &   8 \\ 
##   31 & new jersey &  13 &   8 &   7 &   8 &   3 &   9 &   7 \\ 
##   32 & new mexico &   2 &   4 &   2 &   6 &   5 &   6 &   7 \\ 
##   33 & new york &  26 &  27 &  25 &  23 &  27 &  21 &  24 \\ 
##   34 & north carolina &   3 &   4 &   3 &   3 &   3 &   1 &   2 \\ 
##   35 & north dakota &   0 &   0 &   0 &   0 &   0 &   0 &   0 \\ 
##   36 & ohio &   7 &   3 &   1 &   2 &   3 &   1 &   2 \\ 
##   37 & oklahoma &   1 &   0 &   1 &   1 &   0 &   1 &   5 \\ 
##   38 & oregon &  20 &  28 &  30 &  24 &  26 &  28 &  28 \\ 
##   39 & pennsylvania &   8 &   5 &   4 &   6 &   4 &   6 &   8 \\ 
##   40 & rhode island &   4 &   2 &   3 &   1 &   2 &   0 &   1 \\ 
##   41 & south carolina &   3 &   0 &   0 &   1 &   1 &   0 &   1 \\ 
##   42 & south dakota &   0 &   0 &   0 &   1 &   1 &   0 &   0 \\ 
##   43 & tennessee &   2 &   2 &   2 &   1 &   3 &   1 &   4 \\ 
##   44 & texas &  16 &  14 &  16 &  19 &  12 &  14 &  14 \\ 
##   45 & utah &   1 &   1 &   0 &   2 &   1 &   4 &   4 \\ 
##   46 & vermont &   1 &   2 &   5 &   1 &   3 &   1 &   5 \\ 
##   47 & virginia &   2 &   1 &   5 &   4 &  12 &   1 &   5 \\ 
##   48 & washington &  28 &  22 &  32 &  30 &  32 &  19 &   8 \\ 
##   49 & west virginia &   0 &   0 &   0 &   1 &   1 &   1 &   0 \\ 
##   50 & wisconsin &   2 &   4 &   0 &   2 &   2 &   5 &   5 \\ 
##   51 & wyoming &   0 &   1 &   1 &   1 &   0 &   0 &   0 \\ 
##    \hline
## \end{tabular}
## \end{table}
\end{verbatim}

You can write your entire paper (text, code, analysis, graphics, etc.)
all in R Markdown. As an example, here is a short analysis of the
geographic distribution of Reed College's enrolling students. The
\href{http://www.reed.edu/ir/geographic_states.html}{Institutional
Research Office webpage} has information about the geographic
distribution of Reed's entering classes from 2007-2013.

Figure 1 shows the raw matriculant data from 2013 mapped by state. The
darker a state's shading, the more matriculants from that state. Mousing
over a state will reveal the exact number of students who matriculated
from a certain state.

\begin{figure}[htbp]
\centering
\includegraphics{./reproducible_research_files/figure-latex/unnamed-chunk-3.pdf}
\caption{Domestic Geographic Distribution of 2013 Entering Class}
\end{figure}

However, we may be interested in learning more about the variation in
matriculants across all states rather than identifying the states that
account for the greatest number of matriculants. One way to approach
this task is to map the log of matriculants or to take the log
transformation of the variable of interest. Log transforming a variable
that contains exceptionally large values (i.e., a right skewed variable)
pulls those large values closer to the mean and yields a more
symmetrically distributed variable. As for the map, log transforming the
number of matriculants increases the variation in the color gradient
across states and enables us to better visualize the distribution of
Reed's matriculants across the entire country (as you can see in Figure
2 below).

\begin{figure}[htbp]
\centering
\includegraphics{./reproducible_research_files/figure-latex/unnamed-chunk-4.pdf}
\caption{Domestic Geographic Distribution of 2013 Entering Class (Log
Transformed)}
\end{figure}

\end{document}
